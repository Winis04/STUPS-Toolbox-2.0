\section{Fazit}\raggedbottom
Ziel dieser Bachelorarbeit war es, ein Programm zu erstellen, welches Studenten, die gerade erst mit dem Studium der Theoretischen Informatik begonnen haben, beim Lernen unterstützt. Das tut die im Rahmen dieser Arbeit programmierte Toolbox auch, man muss jedoch darauf achten, wie man sie benutzt. Die Möglichkeit Automaten zu visualisieren und zu bearbeiten ist sicherlich hilfreich, wenn man gerade erst angefangen hat, etwas über sie zu lernen. Zudem ist es auch möglich, einen Automaten Zustand für Zustand und Produktion für Produktion aufzubauen und die \textit{Check String}-Funktion ermöglicht es Schritt für Schritt zu simulieren, wie ein Automat einen Eingabestring verarbeitet. Wenn man jedoch das Programm einfach nur benutzt um Aufgaben zu lösen ohne darüber nachzudenken, ist dies wenig hilfreich für den Lernerfolg.\\
Bei der Simulation von Grammatiken fehlen sicherlich noch einige Funktionen, wie zum Beispiel das Umwandeln einer Grammatik in die Chomsky-Normalform und der CYK-Algorithmus. Da jedoch von vornherein abzusehen war, dass in drei Monaten Entwicklungszeit nur eine begrenzte Anzahl von Funktionen implementiert werden kann, wurde als weiteres Ziel gesetzt, dass die Toolbox einfach erweiterbar sein soll. Dies ist durch das von mir entwickelte Plugin-System gewährleistet. Der modulare Aufbau der grafischen Oberfläche macht die Entwicklung von neuen Plugins flexibel und einfach, ohne zukünftige Entwickler einzuschränken.