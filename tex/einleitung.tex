\section{Einleitung}
\subsection{Motivation}
Dieser Bachelorarbeit liegt eine von Fabian Ruhland im Zuge seiner
Bachelorarbeit im Sommersemester 2016 entwickelte Toolbox \cite{Ruh16} (im
Folgenden als Version 1.0 bezeichnet) zur Visualisierung von Automaten und Grammatiken zu Grunde.
Der Fokus dieser Arbeit lag auf Automaten und der Entwicklung eines Systems, das leicht erweiterbar ist.
Die Toolbox 1.0 stellte au�erdem einige Algorithmen, die f�r die Entwicklung eines Compilers hilfreich sind, zur Verf�gung.


In dieser Arbeit wird die Toolbox nun erweitert. Die Toolbox 1.0 stellte nur einen kleinen Teil der Algorithmen auf dem Gebiet der formalen Sprache zur Verf�gung. Es der Wunsch, diesen Funktionsumfang in dem Ma�e zu vergr��ern, dass die Toolbox als unterst�tzendes Mittel in der Lehre der Vorlesung \textit{Einf�hrung in die theoretische Informatik} eingesetzt werden kann. Au�erdem sollte die Benutzerfreundlichkeit erh�ht werden. 
Die Toolbox soll zum Verst�ndnis der Studierenden beitragen. Hierzu w�re eine nachvollziehbare und schrittweise Durchf�hrung der Algorithmen w�nschenswert.
Es soll Spa� machen, die Toolbox zu benutzen. Ohne viel Aufwand soll es m�glich sein, zu experimentieren und auszuprobieren, wie die Algorithmen sich auswirken.
Zu diesem Zwecke soll die Toolbox eine M�glichkeit zur Verf�gung stellen, �nderungen r�ckg�ngig zu machen.
Eine �berarbeitung der Benutzeroberfl�che soll zu mehr Komfort beitragen.

\subsection{Theoretische Grundlagen}\raggedbottom 
In diesem Abschnitt werden zun�chst grob die theoretischen Grundlagen der formalen Sprachen erl�utert,
die zum Verst�ndnis der Toolbox beitragen. F�r eine genauere Ausf�hrung sei auf \cite{Rot16} verwiesen.

\subsubsection{Alphabet und W�rter}
Ein Alphabet ist eine endliche, nichtleere Menge von Symbolen \cite{Rot16}. Ein Wort �ber einem Alphabet ist eine endliche Kombination aus
Symbolen aus dem Alphabet.
\subsubsection{Formale Sprache}
Eine \textit{formale Sprache} �ber einem Alphabet $\Sigma$ ist eine Menge von W�rtern �ber $\Sigma$.
\subsubsection{Grammatik}
Eine kontextfreie \textit{Grammatik} ist ein Quadrupel $G=(N,\Sigma,S,R)$.
Hierbei sei
\begin{description}
	\item[$\Sigma$] ein Alphabet. Elemente aus diesem Alphabet werden \textit{Terminale} genannt,
	\item[N] eine Menge von Nichtterminalen. Dies sind Symbole, die nicht in $\Sigma$ sind und die durch andere Terminale und Nichtterminale ersetzt werden k�nnen,
	\item[R] eine Menge von \textit{Produktionsregeln}.$P \subset (N \cup \Sigma)^{+} \times (N \cup \Sigma)^{*}$
	\item[S] das \textit{Startsymbol} einer Grammatik, $S \in N$.
\end{description}
Das Anwenden einer Regel wird \textit{Ableiten} genannt und durch $\vdash$ beschrieben. Eine Folge von Nichtterminalen und Terminalen ist eine \textit{Konfiguration}.
Man startet mit dem Startsymbol $S$. Durch Ableiten erh�lt man die n�chste Konfiguration. Durch Ableiten von Symbolen in einer Konfiguration, erh�lt man eine neue Konfiguration. Sobald diese nur noch Terminale enth�lt, wurde ein Wort in der von $G$ erzeugten Sprache gefunden. 

Die durchlaufenen Konfigurationen sind der \textit{Ableitunspfad} dieses Wortes.

Die von G erzeugte Sprache wird mit $L(G)$ bezeichnet.

Eine Grammatik hei�t \textit{kontextfrei}, falls f�r jede Regel $(p,q) \in P$ gilt: $p \in N$.

Die Menge der Sprachen, die von kontextfreien Grammatiken erzeugt werden, ist die
Menge der kontextfreien Sprachen.
\subsubsection{Kellerautomat}
Ein Kellerautomat besteht aus
\begin{itemize}
\item einer Menge von Zust�nden,
\item einem Eingabe-Alphabet,
\item einem Stack,
\item einem Stack-Alphabet,
\item einem initialen Stacksymbol,
\item und einer �berf�hrungsfunktion $\delta$.
\end{itemize}
Ein Kellerautomat befindet sich immer in einem Zustand und bekommt ein Wort als Eingabe. Durch die �berf�hrungsfunktion kann er in einen anderen Zustand �bergehen und dabei ein Symbol der Eingabe lesen und den Stack ver�ndern.
Ein Kellerautomat akzeptiert ein Wort, falls der Stack leer ist, nachdem die Eingabe komplett abgearbeitet wurde.

\subsubsection{�quivalenz}
Zu jeder kontextfreien Grammatik G gibt es einen �quivalenten Kellerautomaten M, d.h. M akzeptiert die von G erzeugte Sprache,
und umgekehrt.





