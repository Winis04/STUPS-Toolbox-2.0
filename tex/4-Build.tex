\section{Build-Prozess}
Als Build-Prozess wird die automatische Erstellung eines fertigen, ausf�hrbaren Programmes bezeichnet.
Bisher wurde dies �ber ein \textit{Makefile} umgesetzt. Dies hatte den Nachteil, dass das Ausf�hren nicht plattformunabh�ngig gewesen ist.
Au�erdem war es unter Anderem auf Grund der externen Bibliotheken nicht m�glich, eine JAR erstellen zu lassen.

Um plattformunabh�ngig zu entwickeln, wurde der Build-Prozess von \textit{Makefiles} auf \textit{Gradle} umgestellt. Gradle\footnote{\url{https://gradle.org/}} ist ein automatisches Build-Tool, welches plattformunabh�ngig ist. Dies erforderte einige Anpassungen des von SableCC erzeugten Codes.
  
  
\subsection{Gradle}\label{Gradle}
Als Build-Tool wurde sich f�r Gradle\footnote{\url{https://gradle.org/}} entschieden. Es ist plattformunabh�ngig und verf�gt �ber die M�glichkeit mit \textit{Groovy}\footnote{\url{http://groovy-lang.org/}}  ausf�hrbare Skripte zu erstellen, was im Folgenden wichtig wird.

F�r die Umstellung musste zun�chst der Aufbau der Packages angepasst werden. Gradle setzt
eine Trennung von Source-Code und Resourcen voraus.
Des Weiteren erwartet Gradle folgende Ordnersturktur: \\

\begin{minipage}[htbp]{\linewidth}
\begin{description}
\item[src]\hfill 
	\begin{description}
	\item[main] \hfill 
		\begin{description}
		\item[java] 
		\item[resource]
		\end{description}
	\item[test] \hfill 
		\begin{description}
			\item[java]
			\item[resource]
			\end{description}
		\end{description}
\end{description}
\end{minipage}\\

Es wurde au�erdem ein weiterer Source-Ordner \textit{GeneratedSource} angelegt, der
die von Gradle generierten Dateien enth�lt.
Gradle sorgt daf�r, dass alle ben�tigten Bibiliotheken heruntergeladen werden
und zum \textit{Java-Classpath} hinzugef�gt werden.
Durch ein Plugin\footnote{\url{https://github.com/johnrengelman/shadow}} kann eine ausf�hrbare .jar-Datei
erzeugt werden, die die importierten Bibliotheken enth�lt. Dies war vorher
nicht m�glich.

\subsubsection{Gradle und SableCC}
Das Zusammenspiel von Gradle und SableCC gestaltete sich nicht einfach. SableCC
ist ein Parser-Generator, welcher von Etienne M. Gagnon entwickelt
wurde\footnote{\url{http://www.sablecc.org/}}. Durch SableCC werden Java-Dateien und Resourcen erzeugt. Leider vermischt SableCC diese.
 Dies hat zur Folge, dass diese Resourcen nicht mehr gefunden werden, da
auf diese mit \textit{getResource} zugegriffen wird und sie sich nicht im
Resource-Ordner befinden, wo Java die Dateien nach Umstellung auf Gradle sucht.

Es war also n�tig, Groovy-Code \ref{lst:build} zu
schreiben, welcher nach dem Build-Prozess die ben�tigten Resourcen in Unterordner des Resource Ordner
von \textit{GeneratedSources} kopiert und anschlie�end die entsprechende
\textit{getResource} Zeilen auf diesen Unterordner anpasst.

\begin{figure}[htb]
\lstinputlisting[frame=single, basicstyle=\small, label=lst:build, caption=Gradle-SableCC-Workaround]{Code/build.gradle}
\end{figure}

Die Toolbox besitzt dabei folgenden Abh�ngigkeiten:
\begin{description}
	\item[JUNG2]\footnote{\url{http://jung.sourceforge.net/}}  Zur Darstellung von Automaten
	\item[JUnit 4]\footnote{\url{http://junit.org/junit4/}} Zum Testen von Code
	\item[Apache Commons-IO]\footnote{\url{http://commons.apache.org/proper/commons-io/}} Zum Bearbeiten von Ordner (wird f�r den Workspace gebraucht)
	\item[Reflections]\footnote{\url{https://github.com/ronmamo/reflections}} Erstellen von Objekten zur Laufzeit, auch in der .jar
	\item[SableCC]\footnote{\url{http://www.sablecc.org/}} Parser-Generator
	\item[Apache Commons Lang]\footnote{\url{https://commons.apache.org/proper/commons-lang/}}  Zum �berschreiben der \textit{equals} und \textit{hashCode}-Methoden
\end{description}

\subsubsection{Build}
Folgende Befehle sind durch die bereitgestellte ausf�hrbare Gradle-Datei aufrufbar:
\begin{itemize}
\item gradle shadowJar \hfill \\
Erstellt eine ausf�hrbare Jar in build/libs.
\item gradle sableCC \hfill \\
Startet SableCC und erstellt die notwendigen Parser. Dieser Befehl ist in dem Befehl \textit{build} enthalten.
\item gradle javadoc \hfill \\
Erstellt die Dokumentation.
\item gradle eclipse \hfill \\
Erm�glicht es, das Programm als Eclipse Projekt zu importieren
\item gradle idea \hfill \\
Erm�glicht es, das Programm als IntelliJ Projekt zu importieren
\item gradle build \hfill \\
Ein kompletter Build.
\end{itemize}
