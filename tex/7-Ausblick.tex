\section{Fazit und Ausblick}\raggedbottom

\subsection{Herausforderungen}
Die Weiterentwicklung eines bereits bestehenden Programmes unterscheidet sich stark von der eigenen Entwicklung eines komplett neuen Programmes.
Obwohl der Aufbau der Arbeit modular ist, war es trotzdem n�tig, sich in den fremden Code einzuarbeiten. Besonders die Umstrukturierung der GUI
machte dies notwendig. Totz der guten Vorarbeit und umfangreichen Dokumentation, 
war der fremde Code nicht immer leicht zu verstehen. Vor Allem ben�tigt das Einarbeiten in Legacy-Code
viel Zeit.
Auch musste man mit Designentscheidungen arbeiten, die man so selber nicht getroffen h�tte.


Der Arbeitsaufwand, welcher neben dem eigentlichen Programmieren anf�llt, wurde zun�chst untersch�tzt und war unerwartet hoch.
Die Umstellung auf Gradle erzwang eine intensive Auseinandersetzung mit g�ngigen Programmierpraktiken,
dazu geh�rte insbesondere die Trennung von Source-Code und Resourcen.
Auch ist es bei so einem gro�en Projekt wichtig, den �berblick zu bewahren. Hierzu geh�rte es, eine gr��ere Kapselung zu erreichen, d.h.
jeder Klasse nur eine Aufgabe zuzuweisen.
Diese war vorher nicht in dem Ma�e gegeben, wie sie jetzt vorhanden ist. Als Beispiel sei hierf�r die Klasse \textit{CLI} genannt, 
die vorher auch die Daten enthielt und verwaltete, was jetzt von der Klasse \textit{Content} �bernomment wurde.

\subsection{Ausblick}
Auf Grund der begrenzten Zeit konnten leider nicht alle w�nschenswerten
Features umgesetzt werden. Da die Toolbox weiterhin modular und leicht
erweiterbar ist, besteht eine F�lle von Funktionen, die noch implementiert
werden k�nnen. Die hervorgehobenen Punkte werden als besonders wichtig erarchtet.
\begin{itemize}
  \item Turing-Maschinen
  \begin{itemize}
    \item Speichern und Laden
    \item \textbf{Visualisierung}
    \item \textbf{animierter Durchlauf}
    \item Algorithmen
    \item Komplexit�tsberechnung
  \end{itemize}
  \item Grammatiken
  \begin{itemize}
    \item Kontextsensitive Grammatiken (erfordert eine Umstrukturierung der
    Klassen ``Rule'' und ``Grammatik' und eine Anpassung des Parsers')
    \item regul�re Ausdr�cke zu Grammatik umwandeln
  \end{itemize}
  \item Sonstiges
  \begin{itemize}
    \item Automaten unmodifizierbar machen
    \item f�r die Algorithmen von Automaten \LaTeX -Ausgabe erzeugen
    \item reine GUI-Funktionen f�r die Konsole umsetzen
    \item \textbf{Programm mit Parametern aufrufen}, so dass die �bergebenen
    Kommandos direkt ausgef�hrt werden k�nnen. Erm�glicht automatisiertes Ausf�hren.
    \item Notizen zu Objekten anlegen
    \item Autovervollst�ndigung von Befehlen
    \item \LaTeX-Generierung: Einstellung, ob \textit{Section}, \textit{Subsection} oder \textit{Paragraph}
  \end{itemize}
\end{itemize}

\subsection{Fazit}
Diese Arbeit setzt die von Herrn Ruhland nahtlos fort. Das Ziel war es, die Toolbox komfortabler in der Benutzung zu machen und
den Funktionsumfang bez�glich Grammatiken zu erweitern. Dies ist gelungen.\hfill \\

Weiterhin gilt jedoch, dass die Toolbox konstruktiv beim Lernen von Grammatiken und Automaten angewendet werden muss.
Die \LaTeX-Funktion und die schrittweise Durchf�hrung der Algorithmen erm�glichen es, ein gr��eres Verst�ndnis
f�r diese zu erlangen. Auch das Berechnen eines Pfades einer Grammatik ist hilfreich.
Auf Grund der recht umfangreichen Features l�sst sie sich leicht missbrauchen; diese Toolbox liefert, bis auf Turing-Maschinen, beinahe alles,
was in der Vorlesung \textit{Theoretische Informatik} im Bereich der formalen Sprachen gelehrt wird.


